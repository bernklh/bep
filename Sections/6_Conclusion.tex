\section{Conclusion}
\label{Chap:Conclusion}
The redesigned stator shows great improvements for both the operational and shut-down condition in the high stress areas. The simulated stress in the shut-down condition is lowered from 2616 [MPa] in the original stator to 1505 [MPa] in the redesign; therefore, the durability of the stator is improved. However, the stress in the shut-down condition still exceeds the yield stress of the stator material. This implies that the blade cracking is not prevented and the desired safety factor, as stated in the RPCs, is not reached. In operating conditions the maximum stress is equal to 810 [MPa], this means that the stator will not crack during operating condition. The safety factor is again not reached for this condition. According to the convergence study, the results obtained in the FEM simulations do agree with the convergence percentage; therefore, the accuracy of the FEM analysis is validated. The newly designed stator retains its functionality and is still compatible within the whole jet engine. Furthermore, the manufacturing plan shows that the redesigned stator is manufacturable using common manufacturing techniques. 

\section{Discussion \& Recommendations}
\label{Chap:DiscussionRecommendations}
Multiple assumptions have been made for the analysis of the stator which led to the results and conclusion mentioned above. The most important assumptions will be discussed in this chapter and some recommendations will be given.\\
\begin{itemize}
\item The estimated temperatures of the operation and shut-down condition are a simplified manner to simulate the effect of these conditions on the stator. Furthermore, shutting down the engine is assumed to happen instantly where only one surface changes temperature. 
\item Only the stresses due to the temperature constraints are taken into account; all other sources of stress are neglected. For example, the stress due to the air flow past the stator is disregarded. In addition, the newly created stator is not ensured of a well controlled airflow as no aerodynamic analysis has been done.
\item In the analysis of the cracking points, the high stresses in the blade-stator transition can be neglected due to the singularities in the FEM software; however in reality, the stator will fail in these points. 
\end{itemize}
These assumptions have influences on the final stress result, although some are minor. These assumptions must be taken into account or proven negligible before the stator can be used in a real engine. So is the temperature distribution, used for the analysis, assumed to be much simpler than it can possibly be. Furthermore, the given temperature constraints are probably chosen more extreme than in reality to ensure that the newly designed stator has a small safety margin. Also the influence of the heat flux between the stator and the surrounding air is not taken into account. 

The recommendations follow out of the discussions, these recommendations can be used to better analyze the stator.
\begin{itemize}
\item An aerodynamic analysis should be applied to the stator to determine whether the new curves and edges have any influence on the airflow. Furthermore, the additional stresses caused by the airflow might prove significant enough to no longer be considered negligible. 
\item With the simulations, the number of experiments can be lowered. However, the final design needs to undergo some experiments to assure its applicability to a real jet engine. The stresses that are neglected can cause the stator to fail and this failure can seriously endanger human lives.
\end{itemize}


\newpage