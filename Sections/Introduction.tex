\section*{Introduction}
\addcontentsline{toc}{section}{Introduction}
\label{introduction}
A manufacturer of glider airplane jet engines identified a problem with its engines. After prolonged testing, it was determined that the cracking of the stator blades was causing the motors to stall. To solve the problem, students of the TU/e were asked to identify the origin of the blade cracking and improve the given stator. The goal for this DBL project is to design an original, but still functional, and compatible re-designed stator which shows great improvements by lowering the stress in the high stress areas.

The given stator is made from the material Inconel-718-aged with a yield stress of 1100 [MPa]. According to the Rankine criteria, the maximum principal stress should not exceed the yield stress in order to prevent cracking \cite{Solidmechanics}. The original stator will be analyzed in two different thermal-mechanical situations: an operational and a shut-down condition. The jet engine's inside temperature rises when operating and cools down again during shut down. Both operating and shut-down cases are investigated as both could cause problems.

For redesigning the stator, a set of design requirements and constraints are set and can be seen in Appendix \ref{AppendixF}. Requirements are set as to provide a target for an acceptable design goal. Constraints are set to ensure the design is feasible. A manufacturing plan is provided to illustrate the methods needed to create the redesigned part. 

The stator is recreated and will be simulated in Siemens NX 10.0. The program enables for geometric, material, dynamical and thermal constraints to be applied. The program makes use of the Finite Element Method (FEM) to run simulations. In order to fully understand this method and its accuracy, a separate assignment is investigated, as is shown in Appendix \ref{AppendixB}. The FEM method makes use of a finite set of elements; this limits the accuracy of the method. A convergence study was conducted to provide a set of criteria that will be used for the stator analysis as can be seen in Appendix \ref{AppendixD}. This criteria will help validate the mesh size. 

In this report, the following topics will be discussed: the validation of the FEM software, an analysis of the current stator, the redesigning of the stator, the results of this newly designed stator, the manufacturing plan of the newly designed stator, and a conclusion on the re-designed stator with a discussion and some recommendations. A list of appendices is provided for more detailed information on certain subjects.
\newpage