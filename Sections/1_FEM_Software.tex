\section{FEM validation}
\label{Chap:fem_software}

\subsection{FEM accuracy}

The stator analysis and redesign makes use of the Finite Element Method. To validate this method, a thin cylindrical disk problem was solved analytically and numerically. The thin disk was chosen because both the radial and tangential stresses are analytically known and can, thus, be compared to the numerical solution. The problem is described in Appendix \ref{AppendixB}.

Due to the functionality of the Finite Element Method, the numerical result should converge towards the analytical solution as the number of elements increase towards infinite. Therefore, the problem is calculated iteratively with increasingly smaller mesh size. This is plotted in Figure \ref{fig:Ass2NumVsAna}.

In this situation, a mesh size smaller than 0.5 [mm] has near no improvement on the numerical result. Once refinements have little effect on the results, the extra computing time required for these refinements will not be worth the slight improvement in accuracy. The difference between analytical and numerical is 2.5 [MPa] here, which is only 0.56$\%$ difference; this means that the accuracy of the FEM simulation is adequate.

\subsection{Mesh refinement}
\label{Meshrefinement}

A satisfying solution can be defined as a point where further mesh refinement has no major effect on the result \cite{meshrefinement}. One way to do this is by calculating the stress in a certain point iteratively until there is no significant change in stress for a smaller mesh size. However, it takes a long time to compute this for every point with high stress because many simulations need to be run. Determining the correct mesh size like this is not very efficient and certainly not always the best solution.

Another method that is used for validation is considering the convergence of the two stresses that FEM simulates: nodal and elemental. During the solution process, in each element, stress results are calculated at certain locations called Gauss points. Nodal stresses in Gauss points can be extrapolated to element nodes. Most often, one node is shared by several elements, and each element reports different stresses at the shared node. Reported values from all adjacent elements are then averaged to obtain a single value. This method of stress averaging produces nodal stress results. 

Alternately, the stress values from all Gaussian points within each element can be averaged to report a single, elemental stress. Although these stresses are averaged between Gauss points, they are called non-averaged 
stresses (or element stresses) because the averaging is done internally within the same element only. 

When refining the mesh, the simulation of the nodal and elemental stresses will, in theory, converge to the same value. The stresses in the areas near the blade did not seem to converge much further than 10$\%$. How this percentage was determined is described in Appendix \ref{AppendixD}. This percentage will be used to determine the correct mesh size throughout the stator validation as it can be applied for much more of the stator than a lower percentage, which is only achieved at areas with lower stresses.
\newpage